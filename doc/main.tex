\documentclass[a4paper,12pt]{article}

\usepackage[utf8]{inputenc}
\usepackage[portuguese]{babel}
\usepackage{graphicx}
\usepackage{hyperref}
\usepackage{listings}
\usepackage{xcolor}
\usepackage{tocbibind} % Para adicionar a bibliografia ao índice
\usepackage{fancyhdr} % Para personalizar os cabeçalhos
\usepackage{geometry}

% Configuração do layout da página
\geometry{top=2cm, bottom=2cm, left=2.5cm, right=2.5cm}

\title{Relatório do Projeto - Estruturas de Dados Avançadas}
\author{Vitor Rezende}
\date{Março de 2025}

\definecolor{codegray}{rgb}{0.5,0.5,0.5}
\definecolor{codeblue}{rgb}{0.25,0.35,0.75}
\definecolor{codegreen}{rgb}{0,0.6,0}
\definecolor{codebackground}{rgb}{0.95,0.95,0.95}

\lstdefinestyle{CStyle}{
    language=C,
    basicstyle=\ttfamily\footnotesize,
    keywordstyle=\color{codeblue}\bfseries,
    stringstyle=\color{codegreen},
    commentstyle=\color{codegray}\itshape,
    backgroundcolor=\color{codebackground},
    numbers=left,
    numberstyle=\tiny,
    stepnumber=1,
    frame=single,
    breaklines=true
}

\pagestyle{fancy}
\fancyhf{}
\fancyhead[L]{Relatório do Projeto EDA}
\fancyhead[C]{Vitor Rezende}
\fancyhead[R]{Março de 2025}

\begin{document}

% Capa
\maketitle
\newpage

% Índice
\tableofcontents
\newpage

% Introdução
\section{Introdução}
Este projeto faz parte da avaliação individual da Unidade Curricular de Estruturas de Dados Avançadas (EDA). O objetivo é reforçar e aplicar conhecimentos adquiridos, especialmente no uso e manipulação de estruturas de dados dinâmicas na linguagem C. Além disso, a implementação exige modularização, armazenamento de dados em ficheiros e documentação utilizando Doxygen.

\newpage

% Contextualização
\section{Contextualização}
O projeto modela uma cidade com várias antenas, cada uma sintonizada numa frequência específica (representada por um caractere). A matriz a seguir ilustra um exemplo de disposição das antenas:

\begin{verbatim}
............
........0...
.....0......
.......0....
....0.......
......A.....
............
............
........A...
.........A..
............
............
\end{verbatim}

Antenas de mesma frequência podem gerar efeitos nefastos (\#), que aparecem em locais específicos dependendo da distância entre as antenas.

\newpage

% Objetivos e Funcionalidades
\section{Objetivos e Funcionalidades}
O projeto deve implementar as seguintes funcionalidades:
\begin{itemize}
    \item Definição de uma estrutura de dados dinâmica (lista ligada) para armazenar as antenas e suas posições.
    \item Leitura dos dados de um ficheiro e armazenamento na estrutura de dados.
    \item Manipulação da lista: inserção e remoção de antenas.
    \item Identificação automática de locais com efeito nefasto e armazenamento desses locais.
    \item Exibição da matriz e dos dados em formato tabular no terminal.
\end{itemize}

\newpage

% Estruturas de Dados
\section{Estruturas de Dados}
Foram utilizadas as seguintes estruturas de dados:

\subsection{Vector2}
\begin{lstlisting}[style=CStyle]
typedef struct Vector2 {
    int x;
    int y;
} Vector2;
\end{lstlisting}

Esta estrutura representa um vetor bidimensional para armazenar coordenadas da matriz.

\subsection{Node}
\begin{lstlisting}[style=CStyle]
typedef struct Node {
    Vector2 pos;
    char value;
    struct Node* next;
} Node;
\end{lstlisting}

Cada nó da lista ligada armazena a posição de uma antena e seu valor (frequência).

\newpage

% Explicação das Funções
\section{Explicação das Funções}

A seguir, são apresentadas as explicações detalhadas de algumas funções importantes do projeto.

\subsection{Função NoiseCheck}
\begin{lstlisting}[style=CStyle]
Node *NoiseCheck(Node *st);
\end{lstlisting}

A função \texttt{NoiseCheck} percorre a lista de antenas e verifica se existe algum efeito nefasto causado pelas antenas. A verificação é realizada com base na comparação de distâncias entre antenas da mesma frequência e, quando uma das antenas está duas vezes mais distante que a outra, aplica-se o efeito nefasto no local correspondente.

\subsection{Função NoiseCheckAlt}
\begin{lstlisting}[style=CStyle]
Node *NoiseCheckAlt(Node *st);
\end{lstlisting}

A função \texttt{NoiseCheckAlt} é uma versão alternativa da \texttt{NoiseCheck}, mas ela compara todas as antenas entre si de forma mais exaustiva. Isso é feito percorrendo a lista inteira e verificando para todas as combinações possíveis de antenas se ocorre o efeito nefasto.

\subsection{Função ReadListFile}
\begin{lstlisting}[style=CStyle]
Node* ReadListFile(const char* filename);
\end{lstlisting}

A função \texttt{ReadListFile} é responsável por carregar os dados das antenas de um ficheiro de texto e armazená-los em uma lista ligada. O ficheiro deve seguir um formato específico onde cada linha representa uma posição da matriz e os caracteres indicam a presença de antenas.

\subsection{Função SaveList}
\begin{lstlisting}[style=CStyle]
void SaveList(const char* filename, Node* st);
\end{lstlisting}

A função \texttt{SaveList} salva a lista de antenas em um ficheiro de texto. Cada nó da lista é percorrido e seus dados (coordenadas e frequência) são gravados no ficheiro de saída.

\newpage

% Resultados e Conclusões
\section{Resultados e Conclusões}
O projeto permite a manipulação de antenas e a análise dos efeitos nefastos de forma eficiente. Com a estrutura dinâmica implementada, é possível adicionar e remover antenas, além de visualizar a matriz e seus impactos. A verificação do efeito nefasto e a atualização da matriz são feitas de forma eficaz e modular.

\newpage

% Bibliografia
\section{Bibliografia}
\begin{itemize}
    \item ISO/IEC 9899:2011, \textit{Programming languages – C}.
    \item Doxygen Documentation. \url{https://www.doxygen.nl/index.html}.
\end{itemize}

\end{document}
